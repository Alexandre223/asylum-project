
\begin{table}[!ht]\centering \footnotesize
\caption{Top 10 source countries}
\begin{tabular}{l c c c c}
\hline \hline
& \multicolumn{2}{c}{First-time applications} & \multicolumn{2}{c}{Asylum decisions}  \\
\textbf{Source country} & \textbf{Total} & \textbf{Share}  & \textbf{Total} & \textbf{Share} \\
\hline 
\smallskip
Russia & 210882 & 7.0\% & 126005 & 5,7\% \\
\smallskip
Iraq & 207447 & 6.9\%  & 173295 & 7.8\% \\
\smallskip
Syria & 181800 & 6.1\%  & 113565 & 5.1\%  \\
\smallskip
Afghanistan & 157124 & 5.2\% & 112370 & 5.1\% \\
\smallskip
Somalia & 137378 & 4.6\% & 84015 & 3.8\%  \\
\smallskip
Iran & 100691 & 3.4\% & 81665 & 3.7\%  \\
\smallskip
Turkey  & 100646 & 3.4\% & 107240 & 4.8\% \\
\smallskip 
Eritrea & 100259 & 3.3\% & 47755 & 2.2\% \\
\smallskip
Serbia & 90869 & 3.0\% & 83485 & 3.8\% \\
\smallskip
Democratic Republic of Congo  & 82829 & 2.8\% &70360 & 3.2\% \\
\hline \hline
\multicolumn{5}{p{12.5cm}}{Note: Column 1 represents the sum of all first-time applications in the 12 European destination countries by citizens of the respective origin country in the years 2002 to 2014. Column 2 represents the share of these first-time applications in all first-time applications in the 12 destination countries from 2002 to 2014. Column 3 shows the number of total asylum decisions for citizens from the respective origin country in the 9 destination countries for which decision data is available. Column 4 shows the respective share of these decisions in all asylum decisions taken in the 9 destination countries between 2002 and 2014. Note that the order of the top 10 origin countries for the decisions is slightly different than that of the applications, as the sample of destination countries differs. Moreover, Eritrea is not in the top 10 of the origin countries in terms of asylum decisions. Instead China is in the top 10 origin countries for asylum decision.}
\end{tabular}
\label{origin_countries}
\end{table}